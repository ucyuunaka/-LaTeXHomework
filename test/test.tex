% ========== 文档类和基本设置 ==========
\documentclass[12pt]{ctexart}
\usepackage{geometry,tikz,graphicx,tabularx,float,xcolor,color,colortbl,cite,booktabs,colortbl}
\usepackage{subcaption,multirow,fancyhdr,gensymb,amsmath,pythonhighlight,multicol,makecell}
\usepackage[hyphens]{url}
\usepackage{seqsplit}
\usepackage[colorlinks,bookmarksopen,bookmarksnumbered,breaklinks=true]{hyperref}
\usepackage{xurl}
% 【注意】使用minted包需要先安装Python和Pygments库,编译时需要在终端或者编辑器设置里添加--shell-escape参数
\usepackage[cache=false]{minted}
\usepackage[T1]{fontenc}

% =============================================
% ========== 【重要】个人信息设置区域 ==========
% =============================================
% 【修改提示】请将下面的"一个学院"等占位符替换为您自己的实际信息
\newcommand{\school}{一个学院}        % 修改为您所在的学院名称,如"计算机学院"
\newcommand{\major}{一个年级专业}     % 修改为您的年级和专业,如"2023级计算机科学与技术"  
\newcommand{\righthead}{一个页眉标题} % 修改为您希望在页眉右侧显示的标题,如"程序设计实验报告"
\newcommand{\maintitle}{一个大标题}   % 修改为您的报告主标题,如"数据结构与算法分析实验报告"
\newcommand{\course}{一个课程}        % 修改为您的课程名称,如"数据结构与算法"
\newcommand{\teacher}{一个老师}       % 修改为您的指导老师姓名
\newcommand{\expnumber}{一个报告类型} % 修改为报告编号或类型,如"实验一"或"期末项目"
\newcommand{\name}{你好、世界、我是、张三、哈哈、你不、认识我啊}          % 修改为您的姓名
\newcommand{\id}{20230001、20230002、20230003、20230004、20230005、20230006、20230007}            % 修改为您的学号

% 【日期设置】以下两行选择一种使用,另一种请注释掉(行首添加%)
% \newcommand{\customdate}{\today}     % 使用当前系统日期(推荐)
\newcommand{\customdate}{2034年2月12日} % 使用自定义日期,修改为您需要的日期格式
% =============================================

% ========== 文档格式设置,一般不需要修改 ==========
% 设置段落首行缩进为2em
\setlength{\parindent}{2em}
% 页面边距设置
\geometry{left=2.54cm,right=2.54cm,top=2.54cm,bottom=2.54cm}
\geometry{headsep=1.2cm}
% 设置页眉页脚样式
\pagestyle{fancy}
\setlength{\headheight}{33pt}
% 设置页眉左侧为校徽图片
\lhead{\includegraphics[width=34.7mm]{figure/badge-horizonal.pdf}}
% 设置页眉右侧标题,使用上面定义的\righthead变量
\rhead{\righthead}

% 定义带下划线的表单项,用于封面信息展示
\makeatletter
\newcommand\dlmu[2][4cm]{\hskip1pt\underline{\hb@xt@ #1{\hss#2\hss}}\hskip3pt}
\makeatother

% 超链接颜色设置和代码块样式设置
\hypersetup{colorlinks=true,linkcolor=black,citecolor=green}
\definecolor{codebg}{rgb}{0.95,0.95,0.95}
% 设置代码块样式:背景色、行号、自动换行
\setminted{
    bgcolor=codebg,
    linenos,
    breaklines,
    breakanywhere,
}

\begin{document}
% 全局使用sloppypar防止文本溢出页面边界
\begin{sloppypar} 

% ========== 封面页设计,一般不需要修改结构 ==========
\begin{titlepage}
    \centering
    % 插入学校校徽,如需更换请替换figure/badge.pdf文件
    \includegraphics[width=9cm]{figure/badge.pdf}
    % 大标题显示,使用上面定义的\maintitle变量
    \\[1cm]\textbf{\huge{\maintitle}}\\[1cm]
    \begin{minipage}{11.1cm}
        \centering
        \begin{flushleft} \Large
            % 以下是表单项,会自动填入上面定义的个人信息
            \makebox[3cm][s]{\textbf{\fangsong{学院:}}}\dlmu[8cm]{\school}\\[0.3cm]
            \makebox[3cm][s]{\textbf{\fangsong{年级专业:}}}\dlmu[8cm]{\major}\\[0.3cm]
            \makebox[3cm][s]{\textbf{\fangsong{课程:}}}\dlmu[8cm]{\course}\\[0.3cm]
            \makebox[3cm][s]{\textbf{\fangsong{指导老师:}}}\dlmu[8cm]{\teacher}\\[0.3cm]
            \makebox[3cm][s]{\textbf{\fangsong{报告编号:}}}\dlmu[8cm]{\expnumber}\\[0.3cm]            
            \makebox[3cm][s]{\textbf{\fangsong{小组成员:}}}\dlmu[8cm]{\name}\\[0.3cm]
            \makebox[3cm][s]{\textbf{\fangsong{学号:}}}\dlmu[8cm]{\id}\\[0.3cm]
            \makebox[3cm][s]{\textbf{\fangsong{日期:}}}\dlmu[8cm]{\customdate}\\[0.3cm]
        \end{flushleft}
    \end{minipage}
    \newpage
\end{titlepage}

% ========== 目录设置 ==========
% 将目录名称设置为中文"目录"
\renewcommand{\contentsname}{目录}
% 目录页使用罗马数字编号(i, ii, iii...)
\pagenumbering{roman} 
% 第一页不显示页码
\thispagestyle{empty}
% 生成目录,自动包含所有section, subsection等
\tableofcontents
\newpage
% 正文开始使用阿拉伯数字页码(1, 2, 3...)
\pagenumbering{arabic} 
% 设置目录中的超链接为红色
\hypersetup{linkcolor=red}

% ==========================================
% ========== 【重要】正文内容开始 ==========
% ==========================================
% 【修改提示】以下是正文示例,您应该根据自己的需求修改或替换这些内容

\section{基本元素示例}

\subsection{文本与段落}
这是一个基本的段落示例。LaTeX会自动处理段落间距和断行。

\subsection{章节与子章节}
文档可以使用\verb|\section{}|、\verb|\subsection{}|和\verb|\subsubsection{}|进行结构化。

\subsection{代码展示}
% 【代码示例】可以根据需要修改为您自己的代码
% 使用minted环境显示代码,{python}指定语言,支持多种编程语言
\begin{minted}{python}
def hello_world():
    print("Hello, World!")
    
if __name__ == "__main__:
    hello_world()
\end{minted}

% 行内代码示例,适合短小的代码片段
行内代码示例:\mintinline{python}{print('Hello')}

\subsection{图片与表格}
\subsubsection{图片示例}
% 【图片插入示例】
% figure环境用于插入图片,[h!]参数尝试将图片放在当前位置
% 将figure/badge-horizonal.pdf替换为您自己的图片路径
\begin{figure}[h!]
    \centering
    \includegraphics[width=0.5\textwidth]{figure/badge-horizonal.pdf}
    \caption{图片示例}  % 图片标题
    \label{fig:example} % 图片标签,用于交叉引用,如\ref{fig:example}
\end{figure}

\subsection{表格示例}
% 使用说明:在LaTeX文档中,表格往往是浮动的,可能不会严格地出现在源代码位置
% 为了确保表格出现在指定位置,推荐使用[H]定位符(所需的float包已导入)
% [H]定位符强制表格精确地放在代码中指定的位置,不会浮动

\subsubsection{三线表}
% 三线表是学术出版物中常用的表格样式,由顶线、中线和底线组成
% 【表格示例1】可以替换为您自己的数据
\begin{table}[H]
    \centering
    \caption{三线表示例}
    \label{tab:example}
    \begin{tabular}{ccc}   % {ccc}表示三列均居中对齐,可使用l(左)、c(中)、r(右)
        \toprule        % 顶线
        \textbf{A} & \textbf{B} & \textbf{C} \\
        \midrule        % 中线
        1 & 2 & 3 \\
        4 & 5 & 6 \\
        \bottomrule     % 底线
    \end{tabular}
\end{table}

\subsubsection{自适应宽度表格}
% 这种表格会自动调整宽度以适应页面,适合包含长文本的表格
% 注意:使用自适应宽度表格需要加载tabularx宏包
% 【表格示例2】可以替换为您自己的数据
\begin{table}[H]
    \centering
    \caption{自适应宽度表格示例}
    \label{tab:adaptive}
    \begin{tabularx}{\textwidth}{|l|X|}  % l表示左对齐,X表示自动调整宽度
        \hline
        \textbf{项目} & \textbf{描述} \\
        \hline
        项目一 & 这是一个自动调整宽度的表格单元格,可以容纳较长的文本内容而不会溢出testtesttesttest \\
        \hline
    \end{tabularx}
\end{table}

\subsection{数学公式}
% 【公式示例】可以替换为您需要的公式
% 行内公式使用$...$包围
行内公式:$E=mc^2$

% 行间公式使用equation环境,会自动编号
行间公式:
\begin{equation}
    F = G\frac{m_1 m_2}{r^2}
\end{equation}

\section{高级功能}
\subsection{超长文本处理}
% 处理超长单词,防止排版溢出
对于超长单词可使用\verb|\seqsplit{}|命令:\seqsplit{supercalifragilisticexpialidocioussupercalifragilisticexpialidocioussupercalifragilisticexpialidocioussupercalifragilisticexpialidocioussupercalifragilisticexpialidocioussupercalifragilisticexpialidocious}

% 长URL示例,会自动换行
长URL示例:\url{https://www.example.com/very/long/path/that/might/cause/problems}

\subsection{网页链接}
% 【链接示例】可以替换为您需要的链接
% 纯网址链接
纯网址链接:\url{https://www.sysu.edu.cn}

% 带自定义文本的链接
自定义文本链接:\href{https://www.sysu.edu.cn}{中山大学官网}

\section{参考文献引用}
\subsection{引用示例}
% 【注意】引用前需确保在main.bib文件中有对应的参考文献条目
% 单一文献引用
单一引用:\cite{knuth1986texbook}

% 多文献同时引用
多文献引用:\cite{vaswani2017attention,brown2020language}

% 带页码的引用
带页码引用:\cite[pp. 45-47]{wang2010zhongwen}

% ==========================================
% ========== 正文内容结束 ==========
% ==========================================

% ========== 参考文献部分 ==========
\newpage
% 将参考文献添加到目录中
\addcontentsline{toc}{section}{参考文献}
% 使用plain引用样式
\bibliographystyle{plain}
% 使用main.bib文件作为参考文献数据库
% 【注意】需要确保main.bib文件存在且包含您引用的所有文献
\bibliography{main}

\end{sloppypar}
\end{document}
