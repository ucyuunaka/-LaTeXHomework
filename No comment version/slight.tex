\documentclass[12pt]{ctexart}
\usepackage{geometry,tikz,graphicx,tabularx,float,xcolor,color,colortbl,cite,booktabs,colortbl}
\usepackage{subcaption,multirow,fancyhdr,gensymb,amsmath,pythonhighlight,multicol,makecell}
\usepackage[hyphens]{url}
\usepackage{seqsplit}
\usepackage[colorlinks,bookmarksopen,bookmarksnumbered,breaklinks=true]{hyperref}
\usepackage{xurl}
\usepackage[cache=false]{minted}
\usepackage[T1]{fontenc}

\newcommand{\school}{一个学院}
\newcommand{\major}{一个年级专业}
\newcommand{\righthead}{一个页眉标题}
\newcommand{\maintitle}{一个大标题}
\newcommand{\course}{一个课程}
\newcommand{\teacher}{一个老师}
\newcommand{\expnumber}{一个报告类型}
\newcommand{\name}{你好、世界、我是、张三、哈哈、你不、认识我啊}
\newcommand{\id}{20230001、20230002、20230003、20230004、20230005、20230006、20230007}

\newcommand{\customdate}{2034年2月12日}

\setlength{\parindent}{2em}
\geometry{left=2.54cm,right=2.54cm,top=2.54cm,bottom=2.54cm}
\geometry{headsep=1.2cm}
\pagestyle{fancy}
\setlength{\headheight}{33pt}
\lhead{\includegraphics[width=34.7mm]{figure/badge-horizonal.pdf}}
\rhead{\righthead}

\usepackage{calc}
\makeatletter
\newcommand\dlmu[2][4cm]{%
  \sbox0{#2}
  \ifdim\wd0>#1\relax
    \hskip1pt\underline{%
      \hb@xt@ #1{%
        \hss\resizebox*{#1}{!}{\mbox{#2}}\hss%
      }%
    }\hskip3pt%
  \else
    \hskip1pt\underline{%
      \hb@xt@ #1{%
        \hss#2\hss%
      }%
    }\hskip3pt%
  \fi
}
\makeatother

\hypersetup{colorlinks=true,linkcolor=black,citecolor=green}
\definecolor{codebg}{rgb}{0.95,0.95,0.95}
\setminted{
    bgcolor=codebg,
    linenos,
    breaklines,
    breakanywhere,
}

\begin{document}
\begin{sloppypar} 

\begin{titlepage}
    \centering
    \includegraphics[width=9cm]{figure/badge.pdf}
    \\[1cm]\textbf{\huge{\maintitle}}\\[1cm]
    \begin{minipage}{11.1cm}
        \centering
        \begin{flushleft} \Large
            \makebox[3cm][s]{\textbf{\fangsong{学院:}}}\dlmu[8cm]{\school}\\[0.3cm]
            \makebox[3cm][s]{\textbf{\fangsong{年级专业:}}}\dlmu[8cm]{\major}\\[0.3cm]
            \makebox[3cm][s]{\textbf{\fangsong{课程:}}}\dlmu[8cm]{\course}\\[0.3cm]
            \makebox[3cm][s]{\textbf{\fangsong{指导老师:}}}\dlmu[8cm]{\teacher}\\[0.3cm]
            \makebox[3cm][s]{\textbf{\fangsong{报告编号:}}}\dlmu[8cm]{\expnumber}\\[0.3cm]            
            \makebox[3cm][s]{\textbf{\fangsong{小组成员:}}}\dlmu[8cm]{\name}\\[0.3cm]
            \makebox[3cm][s]{\textbf{\fangsong{学号:}}}\dlmu[8cm]{\id}\\[0.3cm]
            \makebox[3cm][s]{\textbf{\fangsong{日期:}}}\dlmu[8cm]{\customdate}\\[0.3cm]
        \end{flushleft}
    \end{minipage}
    \newpage
\end{titlepage}

\renewcommand{\contentsname}{目录}
\pagenumbering{roman} 
\thispagestyle{empty}
\tableofcontents
\newpage
\pagenumbering{arabic} 
\hypersetup{linkcolor=red}

\section{基本元素示例}

\subsection{文本与段落}
这是一个基本的段落示例。LaTeX会自动处理段落间距和断行。

\subsection{章节与子章节}
文档可以使用\verb|\section{}|、\verb|\subsection{}|和\verb|\subsubsection{}|进行结构化。

\subsection{代码展示}
\begin{minted}{python}
def hello_world():
    print("Hello, World!")
    
if __name__ == "__main__:
    hello_world()
\end{minted}

行内代码示例:\mintinline{python}{print('Hello')}

\subsection{图片与表格}
\subsubsection{图片示例}
\begin{figure}[h!]
    \centering
    \includegraphics[width=0.5\textwidth]{figure/badge-horizonal.pdf}
    \caption{图片示例}
    \label{fig:example}
\end{figure}

\subsection{表格示例}

\subsubsection{三线表}
\begin{table}[H]
    \centering
    \caption{三线表示例}
    \label{tab:example}
    \begin{tabular}{ccc}
        \toprule
        \textbf{A} & \textbf{B} & \textbf{C} \\
        \midrule
        1 & 2 & 3 \\
        4 & 5 & 6 \\
        \bottomrule
    \end{tabular}
\end{table}

\subsubsection{自适应宽度表格}
\begin{table}[H]
    \centering
    \caption{自适应宽度表格示例}
    \label{tab:adaptive}
    \begin{tabularx}{\textwidth}{|l|X|}
        \hline
        \textbf{项目} & \textbf{描述} \\
        \hline
        项目一 & 这是一个自动调整宽度的表格单元格,可以容纳较长的文本内容而不会溢出testtesttesttest \\
        \hline
    \end{tabularx}
\end{table}

\subsection{数学公式}
行内公式:$E=mc^2$

行间公式:
\begin{equation}
    F = G\frac{m_1 m_2}{r^2}
\end{equation}

\section{高级功能}
\subsection{超长文本处理}
对于超长单词可使用\verb|\seqsplit{}|命令:\seqsplit{supercalifragilisticexpialidocioussupercalifragilisticexpialidocioussupercalifragilisticexpialidocioussupercalifragilisticexpialidocioussupercalifragilisticexpialidocioussupercalifragilisticexpialidocious}

长URL示例:\url{https://www.example.com/very/long/path/that/might/cause/problems}

\subsection{网页链接}
纯网址链接:\url{https://www.sysu.edu.cn}

自定义文本链接:\href{https://www.sysu.edu.cn}{中山大学官网}

\section{参考文献引用}
\subsection{引用示例}
单一引用:\cite{knuth1986texbook}

多文献引用:\cite{vaswani2017attention,brown2020language}

带页码引用:\cite[pp. 45-47]{wang2010zhongwen}

\newpage
\addcontentsline{toc}{section}{参考文献}
\bibliographystyle{plain}
\bibliography{main}

\end{sloppypar}
\end{document}
